\documentclass{article}
\usepackage{tikz}
\usetikzlibrary{shapes,arrows}
\begin{document}
\pagestyle{empty}

% Define block styles
\tikzstyle{decision} = [
	diamond, 
	draw, 
	fill=blue!20, 
	text width=5em, 
	text badly centered, 
	node distance=4cm, 
	inner sep=0pt
	]
\tikzstyle{block} = [
	rectangle, 
	draw, 
	fill=blue!20, 
    text width=20em, 
    text centered, 
    rounded corners, 
    minimum height=4em
    ]
\tikzstyle{line} = [
	draw, 
	-latex'
	]
\tikzstyle{cloud} = [
	draw, 
	ellipse,
	fill=red!20, 
	node distance=4cm,
    minimum height=2em
    ]

\resizebox{!}{20cm}{
	\begin{tikzpicture}[node distance=3cm][scale=0.50]
	    % Place nodes
	    \node [block] (init) {
	    	Pre Registration
	   		};
	    \node [block, below of=init] (requests open) {
	    	Course request through OARS begins (5.1)
	    	};
	    \node [block, below of=requests open] (The phase of unknown) {
	    	Instructors look at your requests and decide to whether to give you the course or not.
	    	};
	    \node [decision, below of=The phase of unknown] (Result of instructors decision) {
	    	Accepted into desired courses!
	    	};
	    \node [block, below of=Result of instructors decision, node distance=4cm] (Now Register) {
	    	Now fill the registeration form , submit it to DUGC and wait for his approval
	    	};
	    \node [decision, below of=Now Register] (whims of DUGC) {
	    	Did the DUGC approve?
	    	};
	    \node [block, below of=whims of DUGC, node distance=4cm] (khushi manao) {
	    	Yay, you got all the courses you wanted to do :)
	    	};
	    \node [block, below of = khushi manao] (courses begin) {
	    	Attend the classes and see if you enjoy them
	    	};
	    \node [decision, below of = courses begin] (The decision period){
	    	Do you enjoy the course?
	    	};
	    \node [block, below of = The decision period,node distance=4cm] (Be Happy) {
	    	Be contended and Do the course for rest of the sem :P
	    	};
	    

	    \node [decision, right of= The phase of unknown, node distance = 8cm] (ab bhi nahi mila) {
	    	Is this is third time trying for same course
	    	};
	  	\node [block, below of = ab bhi nahi mila, node distance=4cm] (Try again) {
	    	No problem go talk to the prof and request course again
	    	};
	    \node [block, right of = ab bhi nahi mila, node distance=8cm] (moh maya) {
	    	Be happy with what you got sabmohmayahain :p
	    	};

	    \node [block, right of=whims of DUGC, node distance=8cm] (Dugc try){
	    	Go talk to DUGC ang convince him.
	    	};

	    \node [block, left of = The phase of unknown, node distance=8cm] (add course) {
	    	Send requests to new courses you want.
	    	};
	    \node [block, left of= Now Register, node distance=8cm] (drop courses) {
	    	Drop the course you don't want.
	    	};
	    \node [block, left of = khushi manao, node distance = 8cm] (add drop) {
	    	Add drop begins
	    };
	    \node [decision, left of = The decision period, node distance=8cm] (No likes given) {
	    	Is add drop period over?
	    	};
	    % Draw edges
	    \path [line] (init) -- (requests open);
	    \path [line] (requests open) -- (The phase of unknown);
	    \path [line] (The phase of unknown) -- (Result of instructors decision);
	    \path [line] (Result of instructors decision) -- node [near start] {yes} (Now Register);
	    \path [line] (Now Register) -- (whims of DUGC);
	    \path [line] (whims of DUGC) -- node{yes}(khushi manao);
	    \path [line] (khushi manao) -- (courses begin);
	    \path [line] (courses begin) -- (The decision period);
	    \path [line] (The decision period) -- node {yes} (Be Happy);
	  	\path [line] (Result of instructors decision) -- node {no}(Try again);
	  	\path [line] (Try again) -- (ab bhi nahi mila);
	  	\path [line] (ab bhi nahi mila) -- node {yes} (moh maya);
	  	\path [line] (ab bhi nahi mila) |- node {no} (requests open);
	  	\path [line] (moh maya) |- (courses begin);
	  	\path [line] (whims of DUGC) -- node {no} (Dugc try);
	  	\path [line] (Dugc try) |- (Now Register);
	  	\path [line] (The decision period) -- (add drop);
	\end{tikzpicture}
}
\end{document}