\documentclass{article}
\usepackage{amsthm}
\usepackage{amsmath}

%%%%%%%%%%%%%%%% Tikz and pgf %%%%%%%%%%%%%%%%%%%%%5
\usepackage{tikz}
\usepackage{pgfmath}
\usetikzlibrary{positioning, arrows,automata}
\usetikzlibrary{shapes}
\usepackage{dot2texi}



\newtheorem{theorem}{Theorem}[section]
\title{Graphs}
\author{Manikanta Reddy}

\begin{document}
  \maketitle
  \section{Peterson's Graph}
    \begin{dot2tex}[neato,mathmode]
      digraph G {
        d2tdocpreamble = "\usetikzlibrary{automata}";
        d2tfigpreamble = "\tikzstyle{every state}= [draw=black!120,very thick,fill=blue!20]";
        node[style= "state"];
        a_1 -> a_2 -> a_3 -> a_4 -> a_5 -> a_1;
        
        b_1 -> b_3; b_2 -> b_4; b_3 -> b_5;  b_4 -> b_1; b_5 -> b_2;

        a_1 -> b_1; 
        a_2 -> b_2;
        a_3 -> b_3;
        a_4 -> b_4;
        a_5 -> b_5;
      }
    \end{dot2tex}

  \section{The Kuratowski graphs}
    \subsection{K(5)}
      \begin{dot2tex}[neato,mathmode]
        digraph G {
          d2tdocpreamble = "\usetikzlibrary{automata}";
          d2tfigpreamble = "\tikzstyle{every state}= [draw=black!120,very thick,fill=blue!20]";
          node[style= "state"];
          1->2; 1->3; 1->4; 1->5;
          2->1; 2->3; 2->4; 2->5;
          3->1; 3->2; 3->4; 3->5;
          4->2; 4->3; 4->1; 4->5;
          5->2; 5->3; 5->4; 5->1;
        }
      \end{dot2tex}
    \subsection{K(3,3)}
      \begin{dot2tex}[neato,mathmode]
        digraph G {
          d2tdocpreamble = "\usetikzlibrary{automata}";
          d2tfigpreamble = "\tikzstyle{every state}= [draw=black!120,very thick,fill=blue!20]";
          node[style= "state"];
          a1->b1; a1->b2; a1->b3;b1->a1; b1->a2; b1->a3;
          a2->b1; a2->b2; a2->b3;b2->a1; b2->a2; b2->a3;
          a3->b1; a3->b2; a3->b3;b3->a1; b3->a2; b3->a3;
        }
      \end{dot2tex}




\end{document}
